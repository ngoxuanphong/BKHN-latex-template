\chapter{GIỚI THIỆU}

Chương 1 nên cung cấp một tuyên bố rõ ràng về vấn đề mà dự án đặt ra và lý do tại sao vấn đề đó lại quan trọng. Nó nên phản ánh tình huống nếu có sẵn. Phần giới thiệu cũng cần cung cấp thông tin nền để người đọc có thể hiểu được sự quan trọng của vấn đề.

\section{Bối cảnh dự án}

\begin{itemize}
\item Tổng quan về tài liệu nghiên cứu: Với mục tiêu cung cấp cái nhìn tổng quan về các nghiên cứu, nghiên cứu và lý thuyết liên quan đến vấn đề hoặc chủ đề được đề cập trong dự án.

\item Sự liên quan và Tầm quan trọng: Tại sao vấn đề này lại quan trọng và tầm quan trọng trong thời điểm hiện tại; Ai sẽ sử dụng giải pháp đề xuất của bạn; Nêu rõ tiềm năng tác động của kết quả dự án của bạn.

\item Tình trạng hiện tại và Hạn chế: Mô tả tình trạng hiện tại hoặc các giải pháp hiện có và nhược điểm của chúng.

\item Chuyển đổi: Kết thúc bằng cách kết nối vào định nghĩa dự án và / hoặc mục tiêu.
\end{itemize}

\section{Định nghĩa dự án}
Xác định vấn đề dựa trên các chi tiết kỹ thuật và tính năng mà sản phẩm, dịch vụ hoặc quy trình cần phải có. Những chi tiết này thường được thiết lập bởi các người hướng dẫn của bạn để đảm bảo dự án thách thức và phù hợp cho một dự án Tốt nghiệp. Dưới đây là cách phân rã nó:

\subsection{Tuyên bố Vấn đề} 
Nêu vấn đề trong một câu hoặc đoạn văn ngắn.

\subsection{Ngữ cảnh và phạm vi}
Xác định phạm vi của vấn đề - những khía cạnh nó bao hàm và những khía cạnh nó không bao hàm.

\subsection{Tầm quan trọng và tác động}
Giải thích tại sao giải quyết vấn đề này lại quan trọng. Có những hậu quả tiềm năng nếu không giải quyết nó?

\subsection{Định lượng (nếu có)}
Nếu có thể, số hóa vấn đề để thể hiện mức độ lớn của nó. Điều này có thể liên quan đến việc trình bày thống kê, dữ liệu hoặc xu hướng liên quan.

Hãy nhớ giữ cho định nghĩa của vấn đề ngắn gọn, tập trung và phù hợp với mục tiêu của dự án Tốt nghiệp của bạn. Nó nên dễ hiểu và cung cấp hướng dẫn rõ ràng cho phần còn lại của đề xuất.

\section{Mục tiêu dự án}
Mục tiêu của một dự án Tốt nghiệp định rõ những mục tiêu cụ thể và kết quả mà bạn và nhóm của bạn định ra để đạt được thông qua dự án. Những mục tiêu này hướng dẫn đội của bạn và cung cấp mục tiêu rõ ràng. Khi viết phần mục tiêu, hãy đảm bảo chúng đáp ứng các yếu tố \textbf{cụ thể, có thể đo lường, thực hiện được, có liên quan và có thời hạn (SMART)}. Dưới đây là một ví dụ:

\begin{itemize}
\item \textbf{Mục tiêu 1}: [Tiêu đề]
      \begin{itemize}
        \item Nêu rõ mục tiêu đầu tiên bằng cách dùng ngôn ngữ rõ ràng.
        \item Mô tả mục tiêu mà bạn dự định đạt được với mục tiêu này.
        \item Đặt ra cụ thể và có thể đo lường.
      \end{itemize}
\end{itemize}

\begin{equation}
E = MC^2
\end{equation}

\section{Các Thông số kỹ thuật dự án}
Dựa trên các mục tiêu dự án ở trên, bạn được yêu cầu cung cấp các yêu cầu kỹ thuật chi tiết, tính năng và đặc điểm mà dự án của bạn phải tuân thủ để đáp ứng mục tiêu của nó. Những yêu cầu này bao gồm các nhu cầu kỹ thuật, kỳ vọng về hiệu suất, các yếu tố thiết kế, tính năng, khả năng tương thích và biện pháp bảo mật. Thông số kỹ thuật hướng dẫn cho quá trình phát triển dự án, đảm

 bảo rằng nó đáp ứng mục tiêu một cách hiệu quả và tuân theo các tiêu chí đã được xác định.

\cite{Tsebook05,SamarakoonTWC13,KeyInforcom2007}.