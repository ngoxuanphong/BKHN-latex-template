\chapter{MÔ TẢ HỆ THỐNG}
Mục đích của Chương 3 là mô tả quá trình thiết kế dự án của bạn. Chi tiết cần đủ để người đọc có thể dễ dàng hiểu được những gì đã được thực hiện. Nên cung cấp một tóm tắt ngắn về phương pháp độc đáo mà nhóm của bạn đã sử dụng để giải quyết vấn đề, cũng bao gồm một giới thiệu ngắn gọn về lý thuyết hoặc các khái niệm được sử dụng để phân tích và tính toán. Để cải thiện sự rõ ràng trong trình bày, phần này có thể được chia thành các phần con như sau:

%%%%%%%%%%%%%%%%%%%%%%%%%%%%%%%%%%%%%%%%%%%%%%%%%%%%%%
\section{Sơ đồ khối của hệ thống}
\begin{itemize}
\item Trình bày tổng quan về kiến trúc của hệ thống thông qua một sơ đồ khối. Mỗi khối nên đại diện cho một thành phần hoặc mô-đun quan trọng của dự án của bạn.
\item Mô tả mục đích và chức năng của mỗi khối, nhấn mạnh mối quan hệ và tương tác giữa chúng.
\end{itemize}

%%%%%%%%%%%%%%%%%%%%%%%%%%%%%%%%%%%%%%%%%%%%%%%%%%%%%%
\section{Thiết kế của mỗi khối và lựa chọn giải pháp tốt nhất}
\begin{itemize}
\item Ở đây, đi sâu vào chi tiết thiết kế của từng khối riêng lẻ từ sơ đồ khối. Thảo luận về các phương án thiết kế khác nhau được xem xét và lý do chọn thiết kế cuối cùng.
\item Nếu có các phương pháp độc đáo được sử dụng, giải thích về chúng. Trình bày cách thiết kế của mỗi khối đóng góp vào chức năng tổng thể của hệ thống.
\end{itemize}

%%%%%%%%%%%%%%%%%%%%%%%%%%%%%%%%%%%%%%%%%%%%%%%%%%%%%%
\section{Kiểm tra của mỗi khối}
\begin{itemize}
\item Phần con này bao gồm các thủ tục kiểm tra toàn diện được tiến hành cho từng khối.
\item Lưu ý rằng phần con này có thể bị loại bỏ/sửa đổi tùy thuộc vào dự án cụ thể.
\end{itemize}

%%%%%%%%%%%%%%%%%%%%%%%%%%%%%%%%%%%%%%%%%%%%%%%%%%%%%%
\section{Triển khai hệ thống}
\begin{itemize}
\item Thảo luận về các bước thực tế được thực hiện để chuyển đổi thiết kế khái niệm thành các thành phần cụ thể. Xem xét các yếu tố như khả năng mở rộng, hiệu suất và triển khai trong thế giới thực được trình bày chi tiết.
\end{itemize}